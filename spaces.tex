\chapter{Przestrzenie}{
    \label{chp:spaces}
    \begin{definition}[Topologia]
        \label{def:topology}
        Nich $X$ będzie zbiorem. \emph{Topologią} w zbiorze $X$ nazywamy rodzinę podzbiorów $\tau\subseteq\mcp(X)$ taką, że:
        \begin{enumerate}[(i)]
            \item $\emptyset, X \in \tau,$
            \item część wspólna dowolnej skończonej liczby zbiorów należących do $\tau$ także należy do $\tau$,
            \item suma dowolnej, nawet nieskończonej liczby zbiorów należących do $\tau$ także należy do $\tau$.
        \end{enumerate}
    \end{definition}
    \begin{definition}[Przestrzeń topologiczna]
        \label{def:topological_space}
        Zbiór $X$ z ustaloną w nim topologią $\tau$ nazywamy \emph{przestrzenią topologiczną} i oznaczamy $(X, \tau)$.
    \end{definition}
    Zbiory należące do $\tau$ nazywa się \emph{otwartymi} w przestrzeni topologicznej $(X, \tau)$ a elementy zbioru $X$ punktami.
    \begin{definition}[Otoczenie punku]
        \label{def:neighbourhood_tological_space}
        Niech $x$ będzie punktem przestrzeni topologicznej $(X, \tau)$. Zbiór $V\subseteq X$ nazywamy \emph{otoczeniem} punktu $x$ gdy istnieje zbiór otwarty $U\in\tau$ taki, że $x\in U\subseteq V.$ 
    \end{definition}
    \begin{definition}[Przestrzeń T\textsubscript{1}]
        \label{def:t_1_space}
        Przestrzeń topologiczną $(X, \tau)$ nazywamy \emph{przestrzenią T\textsubscript{1}}, jeśli dla dowolnych dwóch różnych punktów tej przestrzeni istnieją otoczenia tych punktów nie zawierające drugiego z punktów.  
    \end{definition}
    \begin{definition}[Punkt skupienia zbioru]
        \label{def:limit_point}
        Niech dany będzie zbiór $A$ przestrzeni topologicznej T\textsubscript{1}. \emph{Punktem skupienia zbioru} jest taki punkt $x$ zadanej przestrzeni, dla którego przekrój dowolnego otoczenia punktu $x$ ze zbiorem $A$ jest niepusty.
    \end{definition}
    \begin{definition}[Metryka]
        \label{def:metric}
        Niech $X$ będzie niepustym zbiorem. \emph{Metryką} w zbiorze $X$ nazywamy funkcję $d\colon X\times X\rightarrow [0, +\infty),$ która dla dowolnych $a, b, c\in X$ spełnia warunki:
        \begin{enumerate}[(i)]
            \item niezdegenerowanie tzn. $d(a, b) = 0 \iff a = 0$,
            \item symetria tzn. $d(a, b) = d(b, a),$
            \item nierówność trójkąta tzn. $d(a, b) \leq d(a, b) + d(c, b)$.
        \end{enumerate}            
    \end{definition}
}