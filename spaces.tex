\chapter{Przestrzenie}{
    \label{chp:spaces}
    \begin{definition}[Topologia]
        \label{def:topology}
        Nich $X$ będzie zbiorem. \emph{Topologią} w zbiorze $X$ nazywamy rodzinę podzbiorów $\tau\subseteq\mcp(X)$ taką, że:
        \begin{enumerate}[(i)]
            \item $\emptyset, X \in \tau,$
            \item część wspólna dowolnej skończonej liczby zbiorów należących do $\tau$ także należy do $\tau$,
            \item suma dowolnej, nawet nieskończonej liczby zbiorów należących do $\tau$ także należy do $\tau$.
        \end{enumerate}
    \end{definition}
    \begin{definition}[Przestrzeń topologiczna]
        \label{def:topological_space}
        Zbiór $X$ z ustaloną w nim topologią $\tau$ nazywamy \emph{przestrzenią topologiczną} i oznaczamy $(X, \tau)$.
    \end{definition}
    Zbiory należące do $\tau$ nazywa się \emph{otwartymi} w przestrzeni topologicznej $(X, \tau)$ a elementy zbioru $X$ punktami.
    \begin{definition}[Otoczenie punku]
        \label{def:neighbourhood_tological_space}
        Niech $x$ będzie punktem przestrzeni topologicznej $(X, \tau)$. Zbiór $V\subseteq X$ nazywamy \emph{otoczeniem} punktu $x$ gdy istnieje zbiór otwarty $U\in\tau$ taki, że $x\in U\subseteq V.$ 
    \end{definition}
    \begin{definition}[Przestrzeń T\textsubscript{1}]
        \label{def:t_1_space}
        Przestrzeń topologiczną $(X, \tau)$ nazywamy \emph{przestrzenią T\textsubscript{1}}, jeśli dla dowolnych dwóch różnych punktów tej przestrzeni istnieją otoczenia tych punktów nie zawierające drugiego z punktów.  
    \end{definition}
    \begin{definition}[Punkt skupienia zbioru]
        \label{def:limit_point}
        Niech dany będzie zbiór $A$ przestrzeni topologicznej T\textsubscript{1}. \emph{Punktem skupienia zbioru} jest taki punkt $x$ zadanej przestrzeni, dla którego przekrój dowolnego otoczenia punktu $x$ ze zbiorem $A$ jest niepusty.
    \end{definition}
    \begin{definition}[Metryka]
        \label{def:metric}
        \emph{Metryką} w zbiorze $X$ nazywamy funkcję $d\colon X\times X\rightarrow \mbr,$ która dla dowolnych $x, y, z\in X$ spełnia warunki:
        \begin{enumerate}[(i)]
            \item $d(x, y) = 0 \iff a = 0$ (niezdegenerowanie),
            \item $d(x, y) = d(y, x)$ (symetria),
            \item nierówność trójkąta tzn. $d(x, y) \leq d(x, z) + d(z, y)$ (nierówność trójkąta).
        \end{enumerate}            
    \end{definition}
    Liczba $d(x, y)$ nazywa się odległością pomiędzy punktami $x, y\in X$ w metryce $d$.
    \begin{definition}[Przestrzeń metryczna]
        \label{def:metric_space}
        Zbiór $X$ z ustaloną w nim metryką~$d$ nazywamy \emph{przestrzenią metryczną} i oznaczamy $(X, d)$.
    \end{definition}
    \begin{definition}[Własność Hausdorffa]
        Mówimy, że przestrzeń topologiczna $(X, tau)$ spełnia \emph{własność Hausdorffa} jeżeli dla dowolnych dwóch różnych punktów tej przestrzeni istnieją otoczenia tych punktów, których przecięcie jest zbiorem pustym.
    \end{definition}
    \begin{definition}[Przestrzeń T\textsubscript{2}]
        Przestrzeń topologiczną $(X, \tau)$ nazywamy \emph{przestrzenią T\textsubscript{1}} lub \emph{przestrzenią Hausdorffa}, jeśli spełnia własność Hausdorffa.
    \end{definition}
    \begin{definition}[Kula]
        \label{def:ball}
        \emph{Kulą (otwartą)} w przestrzeni metrycznej $(X, d)$ o środku w punkcie $x_0\in X$ i promieniu $r > 0$ nazywamy zbiór
        \begin{equation}
            B(x_0, r) \colonequals \{x\in X\colon d(x_0, x) < r\}.
        \end{equation}
    \end{definition}
    \begin{statement}
        Niech $(X, d)$ będzie przestrzenią metryczną. Rodzina podzbiorów zbioru $X$:
        \begin{equation}
            \tau(d) \colonequals \{U\subseteq X\colon\nforall{x_0\in U}\nexists{r>0} B(x_0, r)\subseteq U\}
        \end{equation}
        jest topologią w $X$, spełniającą warunek Hausdorffa.
    \end{statement}
}