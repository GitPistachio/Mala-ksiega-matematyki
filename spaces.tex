\chapter{Przestrzenie}{
    \label{chp:spaces}
    \begin{definition}[Topologia]
        \label{def:topology}
        Nich $X$ będzie zbiorem. \emph{Topologią} w zbiorze $X$ nazywamy rodzinę podzbiorów $\tau\subseteq\mcp(X)$ taką, że:
        \begin{enumerate}[(i)]
            \item \label{def:topology_i} $\emptyset, X \in \tau,$
            \item \label{def:topology_ii}część wspólna dowolnej skończonej liczby zbiorów należących do $\tau$ także należy do $\tau$,
            \item \label{def:topology_iii}suma dowolnej, nawet nieskończonej liczby zbiorów należących do $\tau$ także należy do $\tau$.
        \end{enumerate}
    \end{definition}
    \begin{definition}[Przestrzeń topologiczna]
        \label{def:topological_space}
        Zbiór $X$ z ustaloną w nim topologią $\tau$ nazywamy \emph{przestrzenią topologiczną} i oznaczamy $(X, \tau)$.
    \end{definition}
    Zbiory należące do $\tau$ nazywa się \emph{otwartymi} w przestrzeni topologicznej $(X, \tau)$ a elementy zbioru $X$ punktami.
    \begin{definition}[Otoczenie punku]
        \label{def:neighbourhood_tological_space}
        Niech $x$ będzie punktem przestrzeni topologicznej $(X, \tau)$. Zbiór $V\subseteq X$ nazywamy \emph{otoczeniem} punktu $x$ gdy istnieje zbiór otwarty $U\in\tau$ taki, że $x\in U\subseteq V.$ 
    \end{definition}
    \begin{definition}[Przestrzeń T\textsubscript{1}]
        \label{def:t_1_space}
        Przestrzeń topologiczną $(X, \tau)$ nazywamy \emph{przestrzenią T\textsubscript{1}}, jeśli dla dowolnych dwóch różnych punktów tej przestrzeni istnieją otoczenia tych punktów nie zawierające drugiego z punktów.  
    \end{definition}
    \begin{definition}[Punkt skupienia zbioru]
        \label{def:limit_point}
        Niech dany będzie zbiór $A$ przestrzeni topologicznej T\textsubscript{1}. \emph{Punktem skupienia zbioru} jest taki punkt $x$ zadanej przestrzeni, dla którego przekrój dowolnego otoczenia punktu $x$ ze zbiorem $A$ jest niepusty.
    \end{definition}
    \begin{definition}[Metryka]
        \label{def:metric}
        \emph{Metryką} w zbiorze $X$ nazywamy funkcję $d\colon X\times X\rightarrow \mbr,$ która dla dowolnych $x, y, z\in X$ spełnia warunki:
        \begin{enumerate}[(i)]
            \item $d(x, y) = 0 \iff a = 0$ (niezdegenerowanie),
            \item $d(x, y) = d(y, x)$ (symetria),
            \item nierówność trójkąta tzn. $d(x, y) \leq d(x, z) + d(z, y)$ (nierówność trójkąta).
        \end{enumerate}            
    \end{definition}
    Liczba $d(x, y)$ nazywa się odległością pomiędzy punktami $x, y\in X$ w metryce $d$.
    \begin{definition}[Przestrzeń metryczna]
        \label{def:metric_space}
        Zbiór $X$ z ustaloną w nim metryką~$d$ nazywamy \emph{przestrzenią metryczną} i oznaczamy $(X, d)$.
    \end{definition}
    \begin{definition}[Własność Hausdorffa]
        Mówimy, że przestrzeń topologiczna $(X, tau)$ spełnia \emph{własność Hausdorffa} jeżeli dla dowolnych dwóch różnych punktów tej przestrzeni istnieją otoczenia tych punktów, których przecięcie jest zbiorem pustym.
    \end{definition}
    \begin{definition}[Przestrzeń T\textsubscript{2}]
        Przestrzeń topologiczną $(X, \tau)$ nazywamy \emph{przestrzenią T\textsubscript{1}} lub \emph{przestrzenią Hausdorffa}, jeśli spełnia własność Hausdorffa.
    \end{definition}
    \begin{remark}
        Każda przestrzeń T\textsubscript{2} jest również przestrzenią T\textsubscript{1}.
    \end{remark}
    \begin{proof}
        Zależność wynika wprost z definicji przestrzeni T\textsubscript{1} i  T\textsubscript{2}. Jeżeli dla dowolnych dwóch różnych punktów tej przestrzeni istnieją otoczenia tych punktów, których przecięcie jest zbiorem pustym to tym bardziej każde z tych otoczeń nie zawiera drugiego z punktów.  \qed
    \end{proof}
    \begin{definition}[Kula]
        \label{def:ball}
        \emph{Kulą (otwartą)} w przestrzeni metrycznej $(X, d)$ o środku w punkcie $x_0\in X$ i promieniu $r > 0$ nazywamy zbiór
        \begin{equation}
            B(x_0, r) \colonequals \{x\in X\colon d(x_0, x) < r\}.
        \end{equation}
    \end{definition}
    \begin{theorem}
        Niech $(X, d)$ będzie przestrzenią metryczną. Rodzina podzbiorów zbioru $X$:
        \begin{equation}
            \label{eq:metric_topology}
            \tau(d) \colonequals \{U\subseteq X\colon\nforall{x_0\in U}\nexists{r>0} B(x_0, r)\subseteq U\}
        \end{equation}
        jest topologią w $X$, spełniającą warunek Hausdorffa.
    \end{theorem}
    \begin{proof}
        Zacznę od pokazania, że $\tau(d)$ jest topologią udowadniając każdy z warunków z osobna.
        
        Warunek (\ref{def:topology_i}). Zbiór $\emptyset$ z definicji nie zawiera żadnych elementów także warunek by dla każdego punktu tego zbioru istniała kula o środku w tym punkcie zawarta w tym zbiorze jest automatycznie spełniony. Znów zbiór $X$ jest całą przestrzenią także dowolny zbiór w tym dowolna kula w tej przestrzeni jest podzbiorem $X$.
                
        Warunek (\ref{def:topology_ii}). Jeśli $U_1,\ldots,U_n$ należą do $\tau(d)$ oraz $x_0\in U_1\cap\ldots\cap U_n$ i dla każdego $i = 1, \ldots, k$ istnieje promień $r_i > 0$ taki, że $B(x_0, r_i) \subseteq U_i$ to dla $r\colonequals\min\{r_1,\ldots,r_k\}$ zachodzi inkluzja $B(x_0, r)\subseteq U_1\cap\ldots\cap U_n$. 
        
        Warunek (\ref{def:topology_iii}). Jeśli $\{U_i\in\tau(d)\}_{i\in I}$ oraz $x_0\in\bigcup_{i\in I}U_i$ to istnieje~$i_0$ także, że $x_0\in U_{i_0}$ oraz $r$ takie, że $B(x_0, r)\in U_{i_0}\subseteq \bigcup_{i\in I}U_i $, to $\{U_i\}_{i\in I}\in \tau(d)$. Co kończy nam dowód, że $\tau(d)$ jest topologią. 
        
        Kula $B(x, r)$ o środku w $x\in X$ i promieniu $r>0$ należy do $\tau(d).$ Mianowicie dla każdego $y\in B(x, r)$  Niech $x, y$ będą dwoma różnymi punktami w $X$ oraz $r = d(x, y)/2$. Z własności niezdegenerowania oraz dodatniości metryki mamy, że $d(x, y) >0$ także $r> 0$. Wtedy kule $B(x, r), B(y, r)$ są otoczeniami tych punktów ponieważ należą do $\tau(d)$ oraz ich przecięcie jest zbiorem pustym. Załóżmy, że drugie stwierdzenie jest nieprawdziwe, czyli istnieje punkt $z$ należący do obu tych kul, wtedy $d(x, z) < r$ oraz $d(y, z) < r$. $d(x, y) \leq d(x, z) + d(z, y) = d(x, z) + d(y, z) < 2r = d(x, y).$ Co daje sprzeczność i koczy dowód spełnienia własności Hausdorffa jak i cały dowód. \qed
    \end{proof}
    \begin{thmcoro}
        Każda kula $B(x_0, r)$ o środku w $x_0\in X$ i promieniu $r>0$ jest zbiorem otwartym tzn. należy do $\tau(d).$ Co uzasadnia nazywanie kuli kulą otwartą.
    \end{thmcoro}
    \begin{thmcoro}
        Każda przestrzeń metryczna $(X, d)$ z topologią generowaną w zbiorze $X$ prze metrykę $d$ jest przestrzenią T\textsubscript{2}. Oznaczamy ją przez $(X, \tau(d))$.
    \end{thmcoro}
}