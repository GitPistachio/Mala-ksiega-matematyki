\documentclass[a4paper, oneside, 12pt]{report}

\usepackage[MeX]{polski}
\usepackage[utf8]{inputenc}
\usepackage[left=3.0cm, right=2.0cm, top=2.5cm, bottom=2.5cm]{geometry} % ustawienia marginesów
\usepackage{tocloft} % obsługa spisu treści
\usepackage[toc, page]{appendix}
\usepackage{bbm} % zestaw czcionek matematycznych np. do zbioru liczb rzeczywistych, wymiernych itp.
\usepackage{amsmath, amssymb} % rozszerzenie matematycznych środowisk oraz dodatkowe symbole matematyczne
\usepackage[amsmath, thmmarks]{ntheorem} % pozwala na większą kontrolę nad podstawowymi środowiskami matematycznymi jak def. tw. itp
\usepackage{colonequals} % znaki przypisania np. :=, ::=, =: ...
\usepackage{mathrsfs} % dodatkowe symbole, matematyczny alfabet
\usepackage{indentfirst} % pakiet ten sprawia, że do każdego akapitu dodawane jest wcięcie, zgodnie z polskimi standardami drukarskimi.
\usepackage{caption} % obsługa opisów pod tabelami, figurami itp.
\usepackage{enumerate} % rozszerzenie możliwości list
\usepackage[perpage, symbol]{footmisc} % kolekcja sposobów na zamianę składku przypisów. Pierwszy parametr perpage ustawia resetowanie numerowania przypisów per strona, drugi ustawia, że zamiast numerów chcę używać symboli dla kążdych kolenych przypisów

% zdefiniowanie zbioru symboli jakie mają być używane do odnotowywania przypisów
\DefineFNsymbols{dagger}{{$\dagger$}{$\ddagger$}{$\ddagger\dagger$}{$\ddagger\ddagger$}}

% wybór zbioru symboli
\setfnsymbol{dagger}

% interlinia 1.5
\linespread{1.3}

% ustawia francuskie zwyczaje typograficzne (w tym przypadku przestrzen pozostawioną po zakończeniu zdania).
\frenchspacing

% ustawienie numerowania równań zależnego od podsekcji
\numberwithin{equation}{section}

% dodanie kropek w spisie treści
\renewcommand{\cftsecleader}{\cftdotfill{\cftdotsep}}

% definicje struktur theorem, niezbędna paczka amssymb
\makeatletter
\newtheoremstyle{mplain}%
  {\item[\hskip\labelsep \theorem@headerfont ##1\ ##2\theorem@separator]}%
  {\item[\hskip\labelsep \theorem@headerfont ##1\ ##2]\theorem@headerfont (##3)\theorem@separator}
  
\newtheoremstyle{mbreak}%
  {\item[\rlap{\vbox{\hbox{\hskip\labelsep \theorem@headerfont ##1\ ##2\theorem@separator}\hbox{\strut}}}]}%
  {\item[\hskip\labelsep \theorem@headerfont ##1\ ##2]{\theorem@headerfont (##3)\theorem@separator}\newline}
\newtheoremstyle{zplain}%
  {\item[\hskip\labelsep \theorem@headerfont ##1\ ##2\theorem@separator]}%
  {\item[\hskip\labelsep \theorem@headerfont ##1\ ##2\ (##3)\theorem@separator]}
\makeatother

\qedsymbol{$\boxtimes$}

%Definicja
\theoremstyle{plain}
\theoremheaderfont{\normalfont\bfseries}
\theorembodyfont{\normalfont}
\theoremseparator{.}
\theoremindent0cm
\theoremnumbering{arabic}
\theoremsymbol{}
\newtheorem{definition}{Definicja}[chapter]

%Twierdzenie
\theoremstyle{break}
\theoremheaderfont{\normalfont\bfseries}
\theorembodyfont{\normalfont\itshape}
\theoremseparator{.}
\theoremindent0cm
\theoremnumbering{arabic}
\theoremsymbol{}
\newtheorem{theorem}{Twierdzenie}[chapter]

%Stwierdzenie
\theoremstyle{break}
\theoremheaderfont{\normalfont\bfseries}
\theorembodyfont{\normalfont\itshape}
\theoremseparator{.}
\theoremindent0cm
\theoremnumbering{arabic}
\theoremsymbol{}
\newtheorem{statement}{Stwierdzenie}[chapter]

%Dowód
\theoremstyle{nonumberplain}
\theoremheaderfont{\normalfont\itshape}
\theorembodyfont{\normalfont}
\theoremseparator{:}
\theoremindent0cm
\theoremnumbering{arabic}
\theoremsymbol{}
\newtheorem{proof}{Dowód}[chapter]

%Przykład
\theoremstyle{plain}
\theoremheaderfont{\normalfont\bfseries}
\theorembodyfont{\normalfont}
\theoremseparator{.}
\theoremindent0cm
\theoremnumbering{arabic}
\theoremsymbol{}
\newtheorem{example}{Przykład}[chapter]

%Wniosek do twierdzenia
\theoremstyle{break}
\theoremheaderfont{\normalfont\bfseries\itshape}
\theorembodyfont{\normalfont}
\theoremseparator{.}
\theoremindent0cm
\theoremnumbering{arabic}
\theoremsymbol{}
\newtheorem{thmcoro}{Wniosek}[theorem]

%Uwaga
\theoremstyle{break}
\theoremheaderfont{\normalfont\bfseries\itshape}
\theorembodyfont{\normalfont}
\theoremseparator{.}
\theoremindent0cm
\theoremnumbering{arabic}
\theoremsymbol{}
\newtheorem{remark}{Uwaga}[chapter]

% przydane makra
\newcommand{\mcp}{\mathcal{P}} %rodzina zbiorów
\renewcommand{\emptyset}{\text{\O}} % przedefiniowanie zbioru pustego
\renewcommand{\iff}{\Leftrightarrow} % przedefiniowanie symbolu wtedy i tylko wtedy
\renewcommand{\implies}{\Rightarrow} % przedefiniowanie symbolu implikacji
\newcommand{\nforall}[1]{\forall_{#1}\:} % dodanie spacji po symbolu dla każdego

\begin{document}
	% wstawienie spisu treści
	\tableofcontents

    % tymczasowa sekcja zbiorów
     \section{Zbiory}{
    \label{chp:sets}
    \begin{definition}[(Nad)Podzbiór]
        Niech $A, B$ będą zbiorami oraz każdy element zbioru~$A$ jest również elementem zbioru $B$, wtedy $A$ jest \emph{podzbiorem} zbioru $B$, oznaczany przez $A\subseteq B$ lub równoważnie $B$ jest \emph{nadzbiorem} zbioru $A$, oznaczamy przez $B\supseteq A.$
    \end{definition}
    Skrótowo powyższą zależność pomiędzy zbiorami zapisujemy jako $A\subseteq B$.
    \emph{Rodziną zbiorów} będę określać ,,zbiór zbiorów" danego zbioru. Jest to wygodniejsza językowo forma tego określenia pozwalająca uniknąć niezręcznych językowo sytuacji. \emph{Podrodziną} będę określał podzbiór danej rodziny zbiorów (czyli podzbiór zbioru zbiorów).
    \begin{definition}[Zbiór potęgowy]
        \label{def:powerset}
        Niech $X$ będzie zbiorem. Zbiór wszystkich podzbiorów zbioru $X$ nazywamy zbiorem potęgowym i oznaczamy $\mcp(X)$.
    \end{definition}
    
}
     
	% teoria dotycząca szeroko rozumianych przestrzeni bez zagłębiania się w szczegóły elementów do nich należących
	\chapter{Przestrzenie}{
    \label{chp:spaces}
    \begin{definition}[Topologia]
        \label{def:topology}
        Nich $X$ będzie zbiorem. \emph{Topologią} w zbiorze $X$ nazywamy rodzinę podzbiorów $\tau\subseteq\mcp(X)$ taką, że:
        \begin{enumerate}[(i)]
            \item $\emptyset, X \in \tau,$
            \item część wspólna dowolnej skończonej liczby zbiorów należących do $\tau$ także należy do $\tau$,
            \item suma dowolnej, nawet nieskończonej liczby zbiorów należących do $\tau$ także należy do $\tau$.
        \end{enumerate}
    \end{definition}
    \begin{definition}[Przestrzeń topologiczna]
        \label{def:topological_space}
        Zbiór $X$ z ustaloną w nim topologią $\tau$ nazywamy \emph{przestrzenią topologiczną} i oznaczamy $(X, \tau)$.
    \end{definition}
    Zbiory należące do $\tau$ nazywa się \emph{otwartymi} w przestrzeni topologicznej $(X, \tau)$ a elementy zbioru $X$ punktami.
    \begin{definition}[Otoczenie punku]
        \label{def:neighbourhood_tological_space}
        Niech $x$ będzie punktem przestrzeni topologicznej $(X, \tau)$. Zbiór $V\subseteq X$ nazywamy \emph{otoczeniem} punktu $x$ gdy istnieje zbiór otwarty $U\in\tau$ taki, że $x\in U\subseteq V.$ 
    \end{definition}
    \begin{definition}[Przestrzeń T\textsubscript{1}]
        \label{def:t_1_space}
        Przestrzeń topologiczną $(X, \tau)$ nazywamy \emph{przestrzenią T\textsubscript{1}}, jeśli dla dowolnych dwóch różnych punktów tej przestrzeni istnieją otoczenia tych punktów nie zawierające drugiego z punktów.  
    \end{definition}
}
\end{document}
