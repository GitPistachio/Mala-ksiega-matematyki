\section{Zbiory}{
    \label{chp:sets}
    \begin{definition}[(Nad)Podzbiór]
        Niech $A, B$ będą zbiorami oraz każdy element zbioru~$A$ jest również elementem zbioru $B$, wtedy $A$ jest \emph{podzbiorem} zbioru $B$, oznaczany przez $A\subseteq B$ lub równoważnie $B$ jest \emph{nadzbiorem} zbioru $A$, oznaczamy przez $B\supseteq A.$
    \end{definition}
    Skrótowo powyższą zależność pomiędzy zbiorami zapisujemy jako $A\subseteq B$.
    \emph{Rodziną zbiorów} będę określać ,,zbiór zbiorów" danego zbioru. Jest to wygodniejsza językowo forma tego określenia pozwalająca uniknąć niezręcznych językowo sytuacji. \emph{Podrodziną} będę określał podzbiór danej rodziny zbiorów (czyli podzbiór zbioru zbiorów).
    \begin{definition}[Zbiór potęgowy]
        \label{def:powerset}
        Niech $X$ będzie zbiorem. Zbiór wszystkich podzbiorów zbioru $X$ nazywamy zbiorem potęgowym i oznaczamy $\mcp(X)$.
    \end{definition}
    
}