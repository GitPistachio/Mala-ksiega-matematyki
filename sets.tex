\section{Zbiory}{
    \label{chp:sets}
    \begin{definition}[Podzbiór]
        Niech $A, B$ będą zbiorami. Zbiór $A$ nazywamy \emph{podzbiorem} zbioru $B$ jeżeli każdy element $x\in A$ jest jednocześnie elementem $B$. W zapisie logicznym:
        $$A\subseteq B \iff \nforall{x\in A} x\in B.$$
    \end{definition}
    \emph{Rodziną zbiorów} będę określać ,,zbiór zbiorów" danego zbioru. Jest to wygodniejsza językowo forma tego określenia pozwalająca uniknąć niezręcznych językowo sytuacji. \emph{Podrodziną} będę określał podzbiór danej rodziny zbiorów (czyli podzbiór zbioru zbiorów).
    \begin{definition}[Zbiór potęgowy]
        \label{def:powerset}
        Niech $X$ będzie zbiorem. Zbiór wszystkich podzbiorów zbioru $X$ nazywamy zbiorem potęgowym i oznaczamy $\mcp(X)$.
    \end{definition}
}